\documentclass[a4paper, 11pt, BCOR = 0 pt, DIV = 13, oneside]{scrartcl}
\usepackage{tikz}
\usepackage{fancyvrb}
\usetikzlibrary{arrows, shapes}
\usepackage[light, largesmallcaps]{kpfonts}
\usepackage{coffeestains}
\usepackage[useregional]{datetime2}
\usepackage{gitinfo2}
\usepackage{scrlayer-scrpage}
\usepackage{tcolorbox}
\usepackage[french]{babel}
\author{Patrick Bideault}
\usepackage{varioref}
\usepackage{hyperref}
\hypersetup{
	  pdftitle = LaTeX Coffee Stains,
          colorlinks,
	  linkcolor={brown!60!black},
          citecolor={brown!60!black},
          urlcolor={brown!60!black}
        }
\ifoot{\textsf{\textcolor{lightgray}{version \gitRel{} -- \DTMtoday}
  }}
\cfoot{}
\ofoot{\textsf{\thepage}}

\begin{document}

\title{LaTeX Coffee Stains}
\author{une extension de Hanno Rein\\
  entretenue par Patrick Bideault\\
  \texttt{pb-latex@gmx.fr}\\
  ~\\
  \url{https://framagit.org/Pathe/coffeestains}}
\renewcommand{\today}{version \gitRel{} -- \DTMtoday{}}
\maketitle

\coffeestainA{0.9}{0.85}{-5}{6.5cm}{2.3cm}
\label{stainA}
\section{Introduction}
Cette extension offre une fonctionnalité qui manquait à \LaTeX{} depuis
lurette. Elle ajoute des taches de café aux documents. Plutôt que de
d'ajouter manuellement ces taches, un temps précieux est
économisé en les composant directement au sein des documents.

Quatre modèles de taches sont proposés :

\begin{itemize}
\item \verb|\coffeestainA|, une tache ronde formant un arc de $270^\circ$, orné de deux petites
  taches, comme \vpageref[sur la présente page]{stainA} ;
\item \verb|\coffeestainB|, une tache formant un arc de $60^\circ$ environ, comme en
    \vpageref{stainB} ;
\item \verb|\coffeestainC|, deux taches de couleur pâle, visibles en \vpageref{stainC} ;
\item \verb|\coffeestainD|, et une tache double, très colorée, que l'on admirera en \vpageref{stainD}.
\end{itemize}

\section{Utilisation}
Pour utiliser cette extension, il suffit d'enregistrer le document
\texttt{coffeestains.sty} dans le répertoire où vous souhaitez l'utiliser, aux
côté des fichiers \texttt{.tex} que vous désirez orner de taches de café.

Bien évidemment, il est possible de l'installer avec votre distribution \TeX{} ;
pour ce faire, consultez son manuel d'utilisation.

Une fois l'extension prête à l'emploi, ajoutez la ligne suivante dans le
préambule de votre document :
\begin{verbatim}
\usepackage{coffeestains}
\end{verbatim}
\vfill{}

\begin{tcolorbox}
  La tache de café ci-dessus a été obtenue grâce à l'instruction suivante :

  \verb|\coffeestainA{0.9}{0.85}{-5}{6.5cm}{2.3cm}|
\end{tcolorbox}
\newpage{}
\coffeestainB{0.7}{1}{-30}{18 pt}{-135 pt}
\label{stainB}

Pour rehausser une page d'une tache de café, veuillez ajouter au code d'une page
l'une des commandes suivantes :
\begin{verbatim}
\coffeestainA{transparence}{échelle}{angle}{abscisse}{ordonnée}
\coffeestainB{transparence}{échelle}{angle}{abscisse}{ordonnée}
\coffeestainC{transparence}{échelle}{angle}{abscisse}{ordonnée}
\coffeestainD{transparence}{échelle}{angle}{abscisse}{ordonnée}
\end{verbatim}

\begin{itemize}
\item le coefficient de \texttt{tranparence} est $\in [0,1]$;
\item l'\texttt{échelle} est le facteur de proportionnalité ; s'il vaut \texttt{1},
  la tache sera telle que créée par Hanno Rein en 2009 ;
\item l'\texttt{angle} de la tache est relatif au centre de celle-ci ; il est exprimé en
  degrés $\in [0,360]$ ;
\item l'\texttt{abscisse} et l'\texttt{ordonnée} sont exprimées en
  fonction du centre de la page, qui a pour coordonnées $(0, 0)$. Préciser une
  unité de mesure semble être une bonne idée, mais nous ne saurions nous montrer
  trop directifs.
\end{itemize}

\section{Droits d'auteur}
Comme nous ne croyons guère au droit de propriété en ce qui concerne les
créations intellectuelles, il vous est loisible de redistribuer cette extension
comme bon vous semble. Toutes les taches sont faites maison, photographiées par
Hanno Rein et traitées via les logiciels gimp et Inkscape. Les dons ne seront
que de café et sont à adresser à :
\begin{quote}
Hanno Rein\\
University of Toronto at Scarborough\\
DPES Physics and Astrophysics\\
1265 Military Trail\\
Toronto, Ontario M1C 1A4\\
Canada
\end{quote}

\section{Améliorations souhaitables}
L'utilité de cette extension n'est pas à démontrer. Malheureusement, de
nombreuses taches sont toujours ajoutées manuellement sur les documents : taches
de cassoulet, de kimchi, de vin\dots{} à l'avenir, toutes devraient être composées directement.

Par ailleurs, nous ne saurions nous cantonner aux taches que créent les
liquides : combien de documents graisseux voient leurs taches ajoutées
manuellement par des garagistes ? La communauté des utilisateurs de \LaTeX{}
doit s'emparer de ce problème et procurer à ces professionnels les outils qui leur
sont nécessaires.

\vfill{}

\begin{tcolorbox}
  La tache de café ci-dessus a été obtenue grâce à la commande suivante :

  \verb|\coffeestainB{0.7}{1}{-30}{18 pt}{-135 pt}|
\end{tcolorbox}
\newpage{}
\section{Historique des modifications}
\begin{itemize}
\item 3 avril 2009 : tâche initiale accomplie par Hanno Rein. Cette version est
  toujours disponible à l'adresse
  \url{http://legacy.hanno-rein.de/hanno-rein.de/archives/349}. Dans le dépôt
  git actuel, cette version a pour référence 0.1.
\item 23 novembre 2010 : \texttt{coffee2.sty}, une version améliorée fonctionnant
  avec pdflatex. Merci à \href{http://www.sultanik.com/}{Evan Sultanik}! Cette
  version a de nos jours la référence 0.2.
\item 24 mars 2011 : \texttt{coffee3}, une nouvelle version améliorée utilisant
  pdflatex et permettant la mise à l'échelle et la rotation de chacune des
  taches, sans compter le contrôle de leur transparence. Merci au \href{http://pcmap.unizar.es/~pilar/}{professeur Luis
    Randez}! La référence de cette version est dorénavant 0.3.
\item 25 mai 2012 : \texttt{coffee4}, une version encore améliorée, due à
  \href{http://nepsweb.co.uk/homeapr/}{Adrian Robson}, qui écrit : « je constate
  parvenir rarement à placer ma tasse de café au centre exact de mes documents.
  C'est pourquoi j'ai amendé coffee3.sty de manière à permettre la composition
  de taches de café autrement que centrées ». Cette
  version porte désormais la référence 0.4.
\item 1\textsuperscript{er} mai 2021 : le package \texttt{coffeestains} est
  placé sur un dépôt git. La rotation des taches est désormais relative aux
  taches elles-mêmes et non plus au centre de la page. Version 0.5 par Patrick Bideault.
\end{itemize}
\coffeestainC{1}{1}{180}{0}{-5 mm}
\label{stainC}

\section{Apophtegmes et maximes}
\noindent
Grand est le café.

\vspace{5mm}

\noindent
Le café sauvera le monde.

\vfill{}

\begin{tcolorbox}
  La tache de café ornant la présente page a été obtenue grâce à la commande suivante :

  \verb|\coffeestainC{1}{1}{180}{0}{-5 mm}|
\end{tcolorbox}

\begin{tcolorbox}
La page suivante a été enluminée grâce à l'instruction suivante :

  \verb|\coffeestainD{0.4}{0.5}{90}{3 cm}{4 cm}|
\end{tcolorbox}

\newpage{}
\pagestyle{empty}
~\\

\coffeestainD{0.4}{0.5}{90}{3 cm}{4 cm}
\label{stainD}

\vfill{}
\begin{center}
\textsc{C'est intentionnellement que cette page fut laissée vide}

mais il nous a fallu vous le faire savoir.
\end{center}

\vfill{}
\end{document}
