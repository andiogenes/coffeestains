\documentclass[a4paper, 11pt, BCOR = 0 pt, DIV = 11, oneside]{scrartcl}
\usepackage{tikz}
\usepackage{fancyvrb}
\usetikzlibrary{arrows, shapes}
\usepackage{fontspec}
\usepackage{lipsum}
\setmainfont{TeX Gyre Schola}
% \usepackage[math]{anttor}
% \usepackage[no-math]{fontspec}
% \defaultfontfeatures{Ligatures=TeX}
% \setmainfont{Antykwa Torunska}
% \usepackage{polski,anttor}
% \usepackage[QX]{fontenc}
% \usepackage[light, largesmallcaps]{kpfonts}
\usepackage{coffeestains}
\usepackage[useregional]{datetime2}
\usepackage{gitinfo2}
\usepackage{scrlayer-scrpage}
\usepackage{tcolorbox}
\usepackage{varioref}
\usepackage[polish]{babel}
\author{Patrick Bideault}
\usepackage{hyperref}
\hypersetup{
	  pdftitle = LaTeX Coffee Stains,
          colorlinks,
	  linkcolor={brown!60!black},
          citecolor={brown!60!black},
          urlcolor={brown!60!black}
        }
\ifoot{\textsf{\textcolor{lightgray}{version \gitRel{} -- \DTMtoday}
  }}
\cfoot{}
\ofoot{\textsf{\thepage}}

\begin{document}

\title{LaTeX Coffee Stains}
\author{pakiet autorstwa Hanno Rein\\
utrzymywany przez Patricka Bideaulta\\
  \texttt{pb-latex@gmx.fr}\\
  ~\\
  \url{https://framagit.org/Pathe/coffeestains}\\[1 mm]
Dokumentacja przetłumaczona przez Piotra Znajkowskiego}
\renewcommand{\today}{version \gitRel{} -- \DTMtoday{}}
\maketitle

\coffeestainA{0.9}{0.85}{-25}{5.2cm}{0.5cm}
\label{stainA}
\section{Wprowadzenie}
Ten pakiet zapewnia \LaTeX{}-owi podstawową funkcję, której brakowało przez zbyt
długi czas. Dodaje on do dokumentów plamę po kawie. Dużo czasu można
zaoszczędzić, drukując plamy bezpośrednio na stronie, zamiast dodawać je
ręcznie. Można wybierać spośród czterech różnych rodzajów bejc:
\begin{itemize}
\item[A.] $270^\circ$ plama na okręgu z dwoma małymi rozpryskami, jak na tej stronie \vpageref{stainA}
  \item[B.] $60^\circ$ plama na okręgu, jak na następnej stronie \vpageref{stainB}
  \item[C.] dwa rozbryzgi jasnych kolorów, jak na stronie 3 \vpageref{stainC}
  \item[D.] oraz kolorowy podwójny splash, jak na stronie 4 \vpageref{stainD}
\end{itemize}

\section{Zastosowanie}
Aby skorzystać z pakietu, wystarczy umieścić plik \texttt{coffeestains.sty} w katalogu z wszystkimi innymi plikami \texttt{.tex} \emph{lub} zainstalować go prawidłowo (zapoznaj się z podręcznikiem swojej dystrybucji).
Następnie umieść następujący wiersz w nagłówku dokumentu:
\begin{verbatim}
\usepackage{coffeestains}
\end{verbatim}

\vfill{}

\begin{tcolorbox}
  Polecenie użyte do wykonania powyższej plamy kawy to

  \verb|\coffeestainA{0.9}{0.85}{-25}{5cm}{1.3cm}|
\end{tcolorbox}
\newpage{}
\coffeestainB{0.7}{1}{-30}{18 pt}{-135 pt}
\label{stainB}

Aby umieścić plamę po kawie na stronie, umieść jedno z poniższych poleceń w kodzie źródłowym odpowiedniej strony: 
\begin{verbatim}
\coffeestainA{alfa}{skala}{kąt}{odcięta}{rzędna}
\end{verbatim}

\begin{itemize}
\item \texttt{alfa} to współczynnik przezroczystości $\in [0,1]$;
\item \texttt{skala} jest współczynnikiem skali, a standardem jest \texttt{skala}=1;
\item \texttt{kąt} to kąt nachylenia plamy względem jej środka. Jest wyrażony w stopniach $\in [0,360]$;
\item \texttt{odcięta} i \texttt{rzędna} są wyrażone w odniesieniu do środka
  strony, która ma współrzędne (0, 0). Określenie jednostki miary wydaje się
  dobrym pomysłem, ale nie możemy być zbyt rygorystyczni. ale nie możemy być
  zbyt normatywni.
\end{itemize}


\section{Prawo autorskie}
Możesz dowolnie rozpowszechniać ten pakiet, ponieważ nie wierzymy w wymyśloną
własności. Wszystkie plamy zostały wykonane samodzielnie, sfotografowane przez
Hanno Reina, przetworzone za pomocą gimpa i prześledzone za pomocą programu
Inkscape. Darowizny należy przekazywać wyłącznie w kawie na adres
\begin{quote}
Hanno Rein\\
University of Toronto at Scarborough\\
DPES Physics and Astrophysics\\
1265 Military Trail\\
Toronto, Ontario M1C 1A4\\
Canada
\end{quote}

\section{Pożądane usprawnienia}
Pakiet ten jest oczywiście pomocny, ale wiele plam jest nadal ręcznie dodawanych
do dokumentów: plamy po latte, plamy po herbacie, plamy po
gazpacho\dots{} wszystkie one powinny być w przyszłości drukowane.

Nasze wysiłki powinny wykraczać poza plamy z płynów: ile dokumentów jest ręcznie
plamy z tłuszczu są drukowane ręcznie w warsztatach naprawczych? Społeczność
\LaTeX{} powinna zająć się tym problemem i stworzy odpowiednie narzędzia.

\vfill{}

\begin{tcolorbox}
  Polecenie użyte do wykonania powyższej plamy kawy to

  \verb|\coffeestainB{0.7}{1}{-30}{18 pt}{-135 pt}|
\end{tcolorbox}
\newpage{}
\section{Dziennik Zmiany}
\begin{itemize}
\item 3 kwietnia 2009 r.: początkowa plama z kawy autorstwa Hanno Reina. Ta
  wersja jest nadal dostępna na stronie
  \url{http://legacy.hanno-rein.de/hanno-rein.de/archives/349}. W aktualnym
  repozytorium git ta wersja ma teraz oznaczenie 0.1.
\item 23 listopada 2010 r.: \texttt{coffee2.sty}, ulepszona wersja działająca
  z pdflatex. Podziękowania dla \href{http://www.sultanik.com/}{Evana
    Sultanika}! Ta wersja jest teraz oznaczona jako 0.2.
\item 24 marca 2011: \texttt{coffee3}, kolejna ulepszona wersja, która działa
  z pdflatexem i pozwala na skalowanie, obracanie i zmianę przezroczystości
  dowolnej plamy kawy. Podziękowania dla
  \href{http://pcmap.unizar.es/~pilar/}{profesora Luisa Randeza}! Ta wersja ma
  teraz oznaczenie 0.3.
\item 25 maja 2012: \texttt{coffee4}, ulepszona wersja autorstwa
  \href{http://nepsweb.co.uk/homeapr/}{Adriana Robsona}, który pisze: "Rzadko
  udaje mi się odstawić kubek z kawą dokładnie w połowie
  pracy. dokumentów. Dlatego poprawiłem coffee3.sty, aby obsługiwał plamy po
  kawie poza środkiem." Ta wersja wersja ma teraz oznaczenie 0.4.
\item 1 maja 2021 r.: pakiet \texttt{coffeestains} w repozytorium git. Obrót plamek jest
  teraz względem środka samej plamki, a nie środka strony. strony. Wersja 0.5
  autorstwa Patricka Bideaulta.
\end{itemize}
\coffeestainC{1}{1}{180}{0}{-5 mm}
\label{stainC}

\section{Wieczne motta, apoftegmaty i maksymy}

Kawa jest świetna.

\vspace{5mm}

\noindent
Kawa uratuje świat.

\vfill{}

\begin{tcolorbox}
Polecenie użyte do wykonania powyższej plamy kawy to

  \verb|\coffeestainC{1}{1}{180}{0}{-5 mm}|
\end{tcolorbox}

\begin{tcolorbox}
Polecenie użyte w przypadku plamy kawy na następnej stronie to

  \verb|\coffeestainD{0.4}{0.5}{90}{3 cm}{4 cm}|
\end{tcolorbox}

\newpage{}
\pagestyle{empty}
~\\

\coffeestainD{0.4}{0.5}{90}{3 cm}{4 cm}
\label{stainD}

\vfill{}
\begin{center}
\textsc{Ta strona została celowo pozostawiona pusta}

ale musieliśmy ją zepsuć, informując Cię o tym.
\end{center}

\vfill{}
\end{document}
